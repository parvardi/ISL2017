\begin{problem}[Austria Federal Competition for Advanced Students Final Round]
	Determine all composite positive integers $n$ with the following property: If $1 = d_1 < d_2 <
	\ldots < d_k = n$ are all the positive divisors of $n$, then
		\begin{align*}
			(d_2-d_1):(d_3-d_2):\dots : (d_k - d_{k-1}) = 1:2\dots :(k-1).
		\end{align*}
\end{problem}



\begin{problem}[Austria Beginners' Competition]
	Determine all nonnegative integers $n$ having two distinct positive divisors with the same distance from $n/3$.
\end{problem}


\begin{problem}[Azerbaijan Balkan Math Olympiad First TST]
	Find all functions $f:\mathbb{N}\to\mathbb{N}$ such that \[f(f(n))=n+2015,\]
	for all $n\in \mathbb{N}.$ \hfill \href{https://artofproblemsolving.com/community/c6h1200797p5903215}{Link}
\end{problem}

\begin{problem}[Benelux]
	Let $n$ be a positive integer. Suppose that its positive divisors can be partitioned into pairs (i.e. can be split in groups of two) in such a way that the sum of each pair is a prime number. Prove that these prime numbers are distinct and that none of these are a divisor of $n.$ \hfill \href{http://artofproblemsolving.com/community/c6h1236282p6284421}{Link}
\end{problem}



\begin{problem}[CCA Math Bonanza]
	Compute \[\sum_{k=1}^{420} \gcd(k,420).\]
	\flushright \href{http://artofproblemsolving.com/community/c4h1249124p6423895}{Link}
\end{problem}



\begin{problem}[China South East Mathematical Olympiad]
	Let $n$ be a positive integer and let $D_n$ be the set of all positive divisors of $n$. Define \[f(n)=\sum\limits_{d\in D_n}{\frac{1}{1+d}}.\]
	Prove that for any positive integer $m$, \[\sum_{i=1}^{m}{f(i)} <m.\]
	\flushright \href{http://artofproblemsolving.com/community/c6h1281358p6741745}{Link}
\end{problem}



\begin{problem}[China TST]
	Set positive integer $m=2^k\cdot t$, where $k$ is a non-negative integer, $t$ is an odd number, and let $f(m)=t^{1-k}$. Prove that for any positive integer $n$ and for any positive odd number $a\le n$, $\prod_{m=1}^n f(m)$ is a multiple of $a$. \hfill \href{http://artofproblemsolving.com/community/c6h1215111p6043322}{Link}
\end{problem}




\begin{problem}[China TST]
	For any two positive integers $x$ and $d>1$, denote by $S_d(x)$ the sum of digits of $x$ taken in base $d$. Let $a,b,b',c,m$, and $q$ be positive integers, where $m>1,q>1$, and $|b-b'|\ge a$. It is given that there exists a positive integer $M$ such that
	\[S_q(an+b)\equiv S_q(an+b')+c\pmod{m}\]
	holds for all integers $n\ge M$. Prove that the above equation is true for all positive integers $n$. \hfill \href{http://artofproblemsolving.com/community/c6h1217570p6069193}{Link}
\end{problem}




\begin{problem}[Estonia National Olympiad Eleventh Grade]
	Let $n$ be a positive integer. Let $\delta(n)$ be the number of positive divisors of $n$ and let $\sigma(n)$ be their sum. Prove that
	\begin{align*}
	\sigma(n) > \frac{\left(\delta(n)\right)^2}{2}.
	\end{align*}
\end{problem}


\begin{problem}[European Mathematical Cup Juniors]
	Let $d(n)$ denote the number of positive divisors of $n$. For a positive integer $n$ we define $f(n)$ as
	\begin{align*}
	f(n) & = d(k_1) + d(k_2) + \dots + d(k_m),
	\end{align*}
	where $ 1=k_1 < k_2 < \dots < k_m=n$ are all divisors of the number $n$. We call an integer $n>1$ \textit{almost perfect} if $f(n)=n$. Find all almos perfect numbers.
\end{problem}

