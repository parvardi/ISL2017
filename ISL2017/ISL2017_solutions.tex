\begin{problem}[Argentina Intercollegiate Olympiad Second Level]
	Find all positive integers $x$ and $y$ which satisfy the following conditions:
		\begin{enumerate}
			\item $x$ is a $4-$digit palindromic number, and
			\item $y=x+312$ is a $5-$digit palindromic number.
		\end{enumerate}
	\textbf{Note.} A palindromic number is a number that remains the same when its digits are reversed. For example, $16461$ is a palindromic number.
\end{problem}


\begin{problem}[Bundeswettbewerb Mathematik]
	A number with $2016$ zeros that is written as $101010 \dots 0101$ is given, in which the zeros and ones alternate. Prove that this number is not prime.
	\hfill \href{https://artofproblemsolving.com/community/c6h1338112p7252140}{Link}
\end{problem}

\begin{problem}[Caltech Harvey Mudd Math Competition (CHMMC) Fall]
	We say that the string $d_kd_{k-1} \cdots d_1d_0$ represents a number $n$ in base $-2$ if each $d_i$ is either $0$ or $1$, and $n = d_k(-2)^k + d_{k-1}(-2)^{k-1} + \cdots + d_1(-2) + d_0$. For example, $110_{-2}$ represents the number $2$. What string represents $2016$ in base $-2$?
	\hfill \href{https://artofproblemsolving.com/community/c126h1343610p7306481}{Link}
\end{problem}



\begin{problem}[CentroAmerican]
	Find all positive integers $n$ that have $4$ digits, all of them perfect squares, and such that $n$ is divisible by $2, 3, 5$, and $7$. \hfill \href{http://artofproblemsolving.com/community/c6h1259646p6532180}{Link}
\end{problem}



\begin{problem}[Croatia IMO TST, Bulgaria TST]
	Let $p > 10^9$ be a prime number such that $4p + 1$ is also a prime.
	Prove that the decimal expansion of $\frac{1}{4p+1}$ contains all the digits $0,1, \ldots, 9$. \hfill \href{http://artofproblemsolving.com/community/c6h1233196p6246286}{Link}
\end{problem}


\begin{problem}[Germany National Olympiad Second Round Eleventh/Twelfth Grade]
	The sequence $x_1, x_2, x_3, \ldots$ is defined as $x_1 = 1$ and
	\begin{align*}
	x_{k+1} = x_k + y_k \quad \text{for } k=1, 2, 3, \ldots
	\end{align*}
	where $y_k$ is the last digit of decimal representation of $x_k$. Prove that the sequence $x_1, x_2, x_3, \dots$ contains all powers of $4$. That is, for every positive integer $n$, there exists some natural $k$ for which $x_k=4^n$.
\end{problem}



\begin{problem}[Germany National Olympiad Fourth Round Tenth Grade\footnote{Thanks to Arian Saffarzadeh for translating the problem.}]
	A sequence of positive integers $a_1, a_2, a_3, \dots$ is defined as follows: $a_1$ is a $3$ digit number and $a_{k+1}$ (for $k \geq 1$) is obtained by
	\begin{align*}
	a_{k+1} = a_k + 2 \cdot Q(a_k),
	\end{align*}
	where $Q(a_k)$ is the sum of digits of $a_k$ when represented in decimal system. For instance, if one takes $a_1 = 358$ as the initial term, the sequence would be
	\begin{align*}
	a_1 &= 358,\\
	a_2 &= 358 + 2 \cdot 16 = 390,\\
	a_3 &= 390+ 2 \cdot 12 = 414,\\
	a_4 &= 414 + 2 \cdot 9 = 432,\\
	&\phantom{=}\vdots
	\end{align*}
	Prove that no matter what we choose as the starting number of the sequence,
	\begin{enumerate}[(a)]
		\item the sequence will not contain $2015$.
		\item the sequence will not contain $2016$.		
	\end{enumerate}
\end{problem}



\begin{problem}[IberoAmerican]
	Let $k$ be a positive integer and suppose that we are given $a_1, a_2,\dots, a_k$, where $0 \leq a_i \leq 9$ for $i=1,2,\ldots,k$. Prove that there exists a positive integer $n$ such that the last $2k$ digits of $2^n$ are, in the following order, $a_1, a_2,\dots, a_k , b_1, b_2, \dots, b_k$, for some digits $b_1, b_2, \dots, b_k$. \hfill \href{http://artofproblemsolving.com/community/c6h1312214p7032871}{Link}
\end{problem}


\begin{problem}[India IMO Training Camp]
	Given that $n$ is a natural number such that the leftmost digits in the decimal representations of $2^n$ and $3^n$ are the same, find all possible values of the leftmost digit. \hfill \href{http://artofproblemsolving.com/community/c6h1276412p6696449}{Link}
\end{problem}





\begin{problem}[Middle European Mathematical Olympiad]
	A positive integer $n$ is called \textit{Mozart} if the decimal representation of the sequence $1, 2, \ldots, n$ contains each digit an even number of times.
	Prove that:
	\begin{enumerate}
		\item All Mozart numbers are even.
		\item There are infinitely many Mozart numbers.
	\end{enumerate}
	\flushright \href{http://artofproblemsolving.com/community/c6h1295945p6876321}{Link}
\end{problem}

